\documentclass[12pt, letterpaper, titlepage]{article}
\usepackage[letterpaper,top=2cm,bottom=2cm,left=3cm,right=3cm,marginparwidth=1.75cm]{geometry}
\usepackage[T1,T2A]{fontenc}
\usepackage[russian]{babel}
\usepackage[utf8]{inputenc}
\usepackage{amsmath}
\usepackage{parskip}
\usepackage{svg}
\usepackage{array}
\title{"Курсовая работа по дискретной математике"}
\author{Иван Кочкожаров, студент группы М8О-108Б-22}
\begin{document}
\maketitle
\textbf{1.} Определить для орграфа, заданного матрицей смежности $A$ = 
$
\begin{pmatrix}
    0 & 1 & 0 & 1\\
    0 & 0 & 0 & 1\\
    1 & 1 & 0 & 1\\
    0 & 0 & 0 & 0\\
\end{pmatrix}
$
\begin{itemize}
    \item[a)] матрицу односторонней связности;
    \item[б)] матрицу сильной связности;
    \item[в)] компоненты сильной связности;
    \item[г)] матрицу контуров.
\end{itemize}
\underline{\textbf{Решение.}}

\emph{Изображение графа:}
\begin{center}
    \includesvg[scale=0.5]{graphs/graph1}
\end{center}
\emph{Матрица односторнней связности:}
\begin{equation}
A = A(D) = 
\begin{tabular}{ | c | c | c | c | c | c | } 
    \hline
    & $v_{1}$ & $v_{2}$ & $v_{3}$ & $v_{4}$\\ 
    \hline
    $v_{1}$ & 0 & 1 & 0 & 1\\ 
    \hline
    $v_{2}$ & 0 & 0 & 0 & 1\\ 
    \hline
    $v_{3}$ & 1 & 1 & 0 & 1\\
    \hline
    $v_{4}$ & 0 & 0 & 0 & 0\\
    \hline
\end{tabular}.
\end{equation}
$
A^{2} = 
\begin{pmatrix}
    0 & 1 & 0 & 1\\
    0 & 0 & 0 & 1\\
    1 & 1 & 0 & 1\\
    0 & 0 & 0 & 0\\
\end{pmatrix}
\begin{pmatrix}
    0 & 1 & 0 & 1\\
    0 & 0 & 0 & 1\\
    1 & 1 & 0 & 1\\
    0 & 0 & 0 & 0\\
\end{pmatrix}
=
\begin{pmatrix}
    0 & 0 & 0 & 1\\
    0 & 0 & 0 & 0\\
    0 & 1 & 0 & 1\\
    0 & 0 & 0 & 0\\
\end{pmatrix}.
$\\
$
A^{3} = A \cdot A^{2} =
\begin{pmatrix}
    0 & 1 & 0 & 1\\
    0 & 0 & 0 & 1\\
    1 & 1 & 0 & 1\\
    0 & 0 & 0 & 0\\
\end{pmatrix}
\begin{pmatrix}
    0 & 0 & 0 & 1\\
    0 & 0 & 0 & 0\\
    0 & 1 & 0 & 1\\
    0 & 0 & 0 & 0\\
\end{pmatrix}
=
\begin{pmatrix}
    0 & 0 & 0 & 0\\
    0 & 0 & 0 & 0\\
    0 & 0 & 0 & 1\\
    0 & 0 & 0 & 0\\
\end{pmatrix}.
$
\begin{equation}
    \begin{split}
    T(D) = E \lor A \lor A^{2} \lor A^{3} =
    \begin{pmatrix}
        1 & 0 & 0 & 0\\
        0 & 1 & 0 & 0\\
        0 & 0 & 1 & 0\\
        0 & 0 & 0 & 1\\
    \end{pmatrix}
    \lor 
    \begin{pmatrix}
        0 & 1 & 0 & 1\\
        0 & 0 & 0 & 1\\
        1 & 1 & 0 & 1\\
        0 & 0 & 0 & 0\\
    \end{pmatrix}
    \lor &
    \begin{pmatrix}
        0 & 0 & 0 & 1\\
        0 & 0 & 0 & 0\\
        0 & 1 & 0 & 1\\
        0 & 0 & 0 & 0\\
    \end{pmatrix}
    \lor\\
    \lor
    \begin{pmatrix}
        0 & 0 & 0 & 0\\
        0 & 0 & 0 & 0\\
        0 & 0 & 0 & 1\\
        0 & 0 & 0 & 0\\
    \end{pmatrix}
    =
    \begin{pmatrix}
        1 & 1 & 0 & 1\\
        0 & 1 & 0 & 1\\
        1 & 1 & 1 & 1\\
        0 & 0 & 0 & 1\\
    \end{pmatrix}.
    \end{split}
\end{equation}
\emph{Матрица двусторонней связности:}
\begin{equation}
S(D) = T(D) \& [T(D)]^\mathrm{T} =     
\begin{pmatrix}
    1 & 1 & 0 & 1\\
    0 & 1 & 0 & 1\\
    1 & 1 & 1 & 1\\
    0 & 0 & 0 & 1\\
\end{pmatrix}
\&
\begin{pmatrix}
    1 & 0 & 1 & 0\\
    1 & 1 & 1 & 0\\
    0 & 0 & 1 & 0\\
    1 & 1 & 1 & 1\\
\end{pmatrix}
=
\begin{pmatrix}
    1 & 0 & 0 & 0\\
    0 & 1 & 0 & 0\\
    0 & 0 & 1 & 0\\
    0 & 0 & 0 & 1\\
\end{pmatrix}.
\end{equation}
$S(D)=E\Rightarrow$ в графе $D$ нет контуров.

\emph{Компонентны сильной связности:}
\begin{equation}
S_2(D) = S(D) = 
\begin{tabular}{ | c | c | c | c | c | c | } 
    \hline
    & $v_{1}$ & $v_{2}$ & $v_{3}$ & $v_{4}$\\ 
    \hline
    $v_{1}$ & 1 & 0 & 0 & 0\\ 
    \hline
    $v_{2}$ & 0 & 1 & 0 & 0\\ 
    \hline
    $v_{3}$ & 0 & 0 & 1 & 0\\
    \hline
    $v_{4}$ & 0 & 0 & 0 & 1\\
    \hline
\end{tabular}
\end{equation}
$D_1=(V_1,X_1), V_1=\{v_1\}$

$A(D_1)=$
\begin{tabular}{|c|c|}
    \hline
    & $v_{1}$ \\ 
    \hline
    $v_{1}$ & 0 \\ 
    \hline
\end{tabular}\hspace{1cm}$D_1:$\hspace{1cm}
\includesvg[scale=0.09]{graphs/cfc1}

\begin{equation}
    S_2(D) = 
    \begin{tabular}{ | c | c | c | c |  } 
        \hline
        & $v_{2}$ & $v_{3}$ & $v_{4}$ \\ 
        \hline
        $v_{2}$ & 1 & 0 & 0\\ 
        \hline
        $v_{3}$ & 0 & 1 & 0\\
        \hline
        $v_{4}$ & 0 & 0 & 1\\
        \hline
    \end{tabular}
\end{equation}
$D_2=(V_2,X_2), V_2=\{v_2\}$

$A(D_2)=$
\begin{tabular}{|c|c|}
    \hline
    & $v_{2}$ \\ 
    \hline
    $v_{2}$ & 0 \\ 
    \hline
\end{tabular}\hspace{1cm}$D_2:$\hspace{1cm}
\includesvg[scale=0.09]{graphs/cfc2}

\begin{equation}
    S_3(D) = 
    \begin{tabular}{ | c | c | c |  } 
        \hline
        & $v_{3}$ & $v_{4}$ \\ 
        \hline
        $v_{3}$  & 1 & 0\\ 
        \hline
        $v_{4}$  & 0 & 1\\
        \hline
    \end{tabular}
\end{equation}
$D_3=(V_3,X_3), V_3=\{v_3\}$

$A(D_3)=$
\begin{tabular}{|c|c|}
    \hline
    & $v_{3}$ \\ 
    \hline
    $v_{3}$ & 0 \\ 
    \hline
\end{tabular}\hspace{1cm}$D_3:$\hspace{1cm}
\includesvg[scale=0.09]{graphs/cfc3}

\begin{equation}
    S_4(D) = 
    \begin{tabular}{ | c | c | } 
        \hline
        & $v_{4}$ \\ 
        \hline
        $v_{4}$ & 1\\ 
        \hline
    \end{tabular}
\end{equation}
$D_4=(V_4,X_4), V_4=\{v_4\}$

$A(D_4)=$
\begin{tabular}{|c|c|}
    \hline
    & $v_{4}$ \\ 
    \hline
    $v_{4}$ & 0 \\ 
    \hline
\end{tabular}\hspace{1cm}$D_4:$\hspace{1cm}
\includesvg[scale=0.09]{graphs/cfc4}

\emph{Матрица контуров:}
\begin{equation}
    K(D)=A(D) \& S(D) = 
    \begin{pmatrix}
        0 & 1 & 0 & 1\\
        0 & 0 & 0 & 1\\
        1 & 1 & 0 & 1\\
        0 & 0 & 0 & 0\\
    \end{pmatrix}
    \begin{pmatrix}
        1 & 0 & 0 & 0\\
        0 & 1 & 0 & 0\\
        0 & 0 & 1 & 0\\
        0 & 0 & 0 & 1\\
    \end{pmatrix}
    =
    \begin{pmatrix}
        0 & 0 & 0 & 0\\
        0 & 0 & 0 & 0\\
        0 & 0 & 0 & 0\\
        0 & 0 & 0 & 0\\
    \end{pmatrix}
\end{equation}
\end{document}
